
\hl{comentario}
\hlgreen{Aceptado}
E\hlcyan{Idea}
\hlred{Error}
\hlmag{Idea}
\hlblue{otros}

El experimento DUNE (\textit{Deep Underground Neutrino Experiment}) tiene como propósito principal la observación y el estudio detallado de los neutrinos, partículas fundamentales cuya interacción extremadamente débil con la materia, representada por su baja sección eficaz de interacción ($10^{-45} \, \si{\meter^2}$), los hace sumamente difíciles de detectar \cite{Mishra1990}. Sin embargo, estas partículas tienen el potencial de revelar respuestas fundamentales sobre los orígenes y la evolución del universo. En este contexto, el consorcio encargado del Sistema de Detección de Fotones (\textit{PDS}) ha diseñado y desarrollado una plataforma tecnológica de alto rendimiento conocida como \textit{DAPHNE} (\textit{Detector electronics for Acquiring Photons from Neutrinos}). Este sistema desempeña un papel crucial en la adquisición y procesamiento de datos dentro del \textit{Far Detector}, asegurando una recopilación precisa y eficiente de la información científica necesaria, incluso bajo condiciones operativas extremas \cite{Abi2020}.


Como uno de los principales proyectos estratégicos en el ámbito de la física de partículas, DUNE se encuentra respaldado por el Panel de Priorización de Proyectos de Física de Partículas (\textit{P5}), que en su informe de 2023 destacó la importancia de las fases I y II para avanzar en la comprensión de los neutrinos y sus mecanismos de interacción y producción \cite{DUNE_Phase_II}, \cite{P5_Report}. El \textit{Far Detector} de DUNE consiste en seis criostatos que contienen un total de 70 kilotoneladas de argón líquido (\textit{LAr}), situados a 1.5 km bajo la superficie en Sanford, Dakota del Sur. Este entorno plantea retos significativos para el mantenimiento y la reparación, dadas las condiciones ambientales extremas y el acceso limitado a las instalaciones \cite{Abi2020}. 

El sistema \textit{DAPHNE} está compuesto por 150 tarjetas electrónicas, cada una de las cuales integra cinco AFE5808A (\textit{Analog Front End}), diseñados para amplificar, filtrar y digitalizar señales provenientes de los SiPM en las X-Arapucas. Cada tarjeta cuenta con 40 canales de adquisición que operan a una velocidad de muestreo de 62.5 Msps con una resolución de 14 bits, generando un volumen de datos agregado estimado en 450 Gbps. Esto demuestra la capacidad de \textit{DAPHNE} para gestionar la complejidad y los altos requisitos de datos asociados al experimento \hlblue{otros\cite{SystemSpecs2023}}.

Desde su primera implementación en 2019, el sistema \textit{DAPHNE} ha evolucionado a través de tres versiones, cada una desarrollada para satisfacer las crecientes demandas del \textit{PDS}. La arquitectura inicial integró un microcontrolador STM32H753II para el \textit{Slow Control} y una FPGA Artix-7 responsable del \textit{Fast Data Acquisition}. El microcontrolador supervisa parámetros clave, como tensiones, ajustes de ganancia y configuraciones de periféricos, utilizando interfaces estándar como USART, SPI e I2C. Mientras tanto, la FPGA Artix-7 maneja el procesamiento y la transmisión de datos de alta velocidad \cite{STM32H753II}. 

En 2022, la versión 2 de \textit{DAPHNE} introdujo enlaces de alta velocidad basados en Gigabit Ethernet, gestionados mediante el protocolo TCP/IP. En 2023, la versión 3 presentó una arquitectura optimizada y compacta, utilizando un \textit{System on Module} Kria K26, que incorpora un coprocesador ARM y una FPGA embebida, mejorando significativamente la eficiencia del sistema \cite{DUNE_Far_Detector}. 

El ambiente de operación en Sanford representa desafíos técnicos considerables, como temperaturas extremas, alta humedad, exposición a radiación y acceso limitado a las cavernas subterráneas. Adicionalmente, los protocolos de prueba actuales de \textit{DAPHNE} no están completamente estandarizados, lo que genera una dependencia de operadores especializados y puede derivar en errores sistemáticos. Estas dificultades prolongan los tiempos de validación y aumentan los costos, poniendo en riesgo el cronograma del experimento \cite{Esteban_Ferrer}.

Para superar estas limitaciones, el Grupo de Instrumentación Científica y Microelectrónica (\textit{GICM}) de la Universidad de Antioquia está desarrollando un banco de pruebas automatizado, diseñado bajo protocolos de medición estandarizados. Este sistema tiene como objetivo optimizar los tiempos de prueba, minimizar las inconsistencias y mitigar riesgos de implementación. Además, se plantea la integración de herramientas de inteligencia artificial para analizar patrones en las condiciones operativas de las tarjetas \textit{DAPHNE}, lo que permitirá implementar estrategias de mantenimiento preventivo, reducir la incidencia de fallas y minimizar la intervención humana \hlblue{otros} \cite{GICMDevelopment2023}.

El banco de pruebas permitirá realizar mediciones de parámetros críticos, como los voltajes de operación, las impedancias de las tarjetas y el desempeño de los sistemas de digitalización y transmisión de señales. Este artículo se centra en los avances logrados en la evaluación de los sistemas de digitalización, enfocándose en la medición de la frecuencia de operación de los subsistemas de \textit{DAPHNE}, evaluando su estabilidad y confiabilidad mediante el uso de matrices de puertas lógicas programables (\textit{FPGA}), herramientas clave para asegurar el éxito del experimento DUNE.